\section{Bevezetés}

\indent A mai rohanó világban egyre többen választják az autót a tömegközlekedés helyett. Személy szerint mindig is a tömegközlekedést részesítettem előnyben, ha tehettem – kényelmesebb, fenntarthatóbb megoldás, és sokszor olcsóbb is, mint autóval közlekedni. Ugyanakkor számos alkalommal szembesültem azzal, hogy a buszközlekedés nem mindig olyan egyszerű, mint amilyennek elsőre tűnik. Például Marosvásárhelyen több ízben is hosszú időt töltöttem azzal, hogy rájöjjek, egy adott járat pontosan honnan indul, hol áll meg, és hogyan jutok el egyik pontról a másikra.

Gyakran előfordult, hogy a buszmegállók nevei nem egyeztek az utcák nevével, így a térképen sem lehetett őket pontosan beazonosítani. Ráadásul az útvonalak sokszor nincsenek térképesen megjelenítve, és a menetrendek is nehezen értelmezhetőek. A legtöbb online elérhető felület túlzsúfolt, nem felhasználóbarát, így ahelyett, hogy segítséget nyújtana, inkább további zavart kelt.

Ezek a személyes tapasztalatok ösztönöztek arra, hogy elindítsam ezt a projektet, amelynek célja egy olyan webes alkalmazás megalkotása, amely segít a felhasználóknak eligazodni a buszközlekedés világában. A rendszer egyszerű, letisztult kezelőfelületet biztosít, és térképes nézettel segíti a járatok és útvonalak átláthatóságát.

A fejlesztés során nemcsak az utasokra gondoltam, hanem a szolgáltatók, például a sofőrök, diszpécserek és céges adminisztrátorok igényeit is figyelembe vettem. Mivel régóta érdekel a vállalatirányítás és az ügyfélkapcsolat-kezelés (CRM), fontosnak tartottam olyan funkciók beépítését is, amelyek támogatják a belső munkafolyamatokat, és hatékonyabbá teszik a szervezeti működést.

A projekt webes felületen működik, mivel napjainkban ez a legelérhetőbb és legpraktikusabb forma a felhasználók számára. A .NET technológia mellett döntöttem, mivel számos modern funkcióval, stabil háttérrel és vizuálisan igényes UI csomaggal rendelkezik, amelyekkel öröm dolgozni. A rendszer így nemcsak funkcionálisan, hanem megjelenésében is a mai igényekhez igazodik.

Végezetül fontosnak tartom megjegyezni, hogy ez a projekt nem egy lezárt megoldás, hanem egy olyan alap, amely később számos irányba továbbfejleszthető. Legyen szó mobilalkalmazásról, mesterséges intelligencia alapú útvonaltervezésről vagy mélyebb CRM-rendszer beépítéséről, a lehetőségek széles skálája áll nyitva.

