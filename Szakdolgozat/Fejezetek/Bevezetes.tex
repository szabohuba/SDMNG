\section{Bevezetés}

\indent A mai rohanó világban egyre többen választják az autót a tömegközlekedés helyett. Személy szerint mindig is a tömegközlekedést részesítettem előnyben, ha tehettem – kényelmesebb, fenntarthatóbb megoldás, és sokszor olcsóbb is, mint autóval közlekedni. Ugyanakkor sok alkalommal szembesültem azzal, hogy a tömegközlekedés nem mindig olyan egyszerű, mint amilyennek az elsőre tűnik. Például Marosvásárhelyen több ok miatt is hosszú időt töltöttem azzal, hogy rájöjjek, egy adott járat pontosan honnan indul, hol áll meg, és hogyan jutok el egyik pontról a másikba.

Gyakran előfordult, hogy a buszmegállók nevei nem egyeztek meg az utcák nevével, így a térképen sem lehetett őket pontosan beazonosítani. Ráadásul az útvonalak sokszor nincsenek térképesen megjelenítve, vagy egy oldalon ábrázolva a megállók neveivel, valamint a menetrendek is nehezen értelmezhetőek. A legtöbb online elérhető felület túlzsúfolt, nem felhasználóbarát, így ahelyett, hogy segítséget nyújtana, inkább további zavart keltett bennem.

Ezek a személyes tapasztalatok sarkalltak arra, hogy elindítsam ezt a projektet, amelynek célja egy olyan webes alkalmazás létrehozása, amely segít a felhasználóknak eligazodni a buszközlekedés világában, a fenti problémákat korrigálva.

\indent Az alkalmazás tervezésekor számos funkciót fogalmaztam meg, amelyek célja, hogy a felhasználói élményt javítsák, valamint az információk könnyen hozzáférhetőek és átláthatóak legyenek. Az egyik fő funkció egy olyan külön oldal kialakítása, amely kizárólag egy adott megálló részletes adatait tartalmazza. Az oldalon megjelenik a megálló neve, pontos címe, valamint egy térképes nézet, amely kizárólag az adott megállót mutatja be. A megálló helyzete pontosan lokalizálható lesz, és a cím is egyértelműen olvasható formában szerepel.

Továbbá, az útvonalak megjelenítésére egy dinamikus, interaktív térképet tervezek, amely lehetőséget nyújt a nagyításra és referencia pontokkal való keresésre. Ezáltal a felhasználók könnyen tájékozódhatnak az egyes útvonalak között, és vizuálisan is átláthatják a busz haladásának sorrendjét. A megállók elhelyezése, valamint az útvonalak sorrendje világosan feltüntetésre kerül, elősegítve az egyszerűbb útvonaltervezést.

A jegyvásárlási rendszer újragondolására is sor kerül. A jegyosztály kibővített formában fogja tartalmazni az összes lényeges információt: a vásárlás idejét, az indulási és érkezési időpontokat, a járatot teljesítő busz rendszámát, az adott ülőhely számát, valamint a jegy tulajdonosának adatait. Ezen felül itt is biztosított lesz a dinamikus térképes megjelenítés lehetősége, amelyen a választott útvonal és megállók vizualizálása történik.

A fejlesztés során nemcsak az utasokra gondoltam, hanem a szolgáltatók, például az alkalmazottak és céges adminisztrátorok igényeit is figyelembe vettem. Mivel régóta érdekel a vállalatirányítás és az ügyfélkapcsolat-kezelés (CRM), fontosnak tartottam olyan funkciók beépítését is, amelyek támogatják a belső munkafolyamatokat, és hatékonyabbá teszik a szervezeti működést.

A projekt webes felületen működik, mivel napjainkban ez a legelérhetőbb és legpraktikusabb forma a felhasználók számára. A .NET technológia mellett döntöttem, mivel számos modern funkcióval, stabil háttérrel és vizuálisan igényes UI csomaggal rendelkezik. A rendszer így nemcsak funkcionálisan, hanem megjelenésében is a mai igényekhez igazodik.

Végezetül fontosnak tartom megjegyezni, hogy ez a projekt nem egy lezárt megoldás, hanem egy olyan alap, amely később számos funkcionalitással továbbfejleszthető. Legyen szó mobilalkalmazásról, mesterséges intelligencia alapú útvonaltervezésről vagy részletesebb CRM-rendszer beépítéséről, a lehetőségek széles skálája áll nyitva.


