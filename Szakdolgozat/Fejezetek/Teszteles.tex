\section{Az alkalmazás tesztelése}

A fejlesztési folyamat szerves része a különféle tesztelési technikák alkalmazása, hiszen ezek segítségével biztosítható, hogy az elkészült rendszer megbízhatóan működjön a valós felhasználás során is. A tesztelés nem csupán a hibák kiszűrésére szolgál, hanem hozzájárul a kód minőségének növeléséhez és az esetleges regressziók elkerüléséhez is. Az alábbi alfejezetekben bemutatom a különböző tesztelési megközelítéseket, majd konkrét példán keresztül ismertetem, hogyan történt a projektben az automatizált tesztelés, végül pedig a felhasználói tesztelés tapasztalatait is összefoglalom.

\subsection{Tesztelési módszerek és gyakorlatok a fejlesztés során}

A webalkalmazás megbízható működésének biztosítása érdekében a fejlesztés során különös figyelmet fordítottam a rendszer több szintű és különböző szemléletű tesztelésére. A szoftvertesztelési technikák alapvetően három fő megközelítés köré csoportosíthatók: black-box, white-box és gray-box (silver-box) tesztelés.

A black-box tesztelés lényege, hogy a rendszer működését kizárólag annak bemenetei és kimenetei alapján vizsgáljuk, anélkül hogy a belső logikát ismernénk. Ez a módszer jól alkalmazható felhasználói szemszögből, mivel az alkalmazás viselkedése alapján értékeli az elvárt működést. Ezzel szemben a white-box tesztelés a fejlesztői nézőpontot képviseli: itt a belső algoritmusok, logikai elágazások és adatszerkezetek ismerete alapján történik az ellenőrzés. A gray-box tesztelés pedig e két megközelítés kombinációját képviseli, részleges rálátással a rendszer szerkezetére, ugyanakkor külső szemlélőként történő értékeléssel.

A projekt megvalósítása során mindhárom típusú tesztelést alkalmaztam. A fejlesztési ciklus korai és középső szakaszában manuális, white-box alapú ellenőrzéseket végeztem. Ezek során az újonnan implementált funkciók működését közvetlenül, a kódszintű logikát is figyelembe véve ellenőriztem. Ennek keretében minden egyes funkció kipróbálásra került különböző bemeneti adatokkal és szélssőséges esetekkel, ezzel segítve a korai hibafelismerést.

Ezt követően az alkalmazás kritikus részei – különösen a kontroller osztályok – automatizált egységtesztelésen estek át xUnit keretrendszer segítségével. Ezek a tesztek lehetővé tették, hogy programozott módon ellenőrizzem például azt, hogy a kontroller visszaad-e egy helyes nézetet érvényes vagy érvénytelen azonosító esetén, illetve hogy megfelelő hibaüzenetek keletkeznek-e bizonyos feltételek esetén. Az egységtesztek előnye, hogy gyorsan futtathatók, jól integrálhatók a fejlesztési folyamatba, és az alkalmazás belső működésére koncentrálnak.

A harmadik, és a végső szakaszban alkalmazott tesztelési forma a felhasználói szintű tesztelés volt, amely során valós felhasználók próbálták ki az alkalmazást különböző eszközökön és környezetekben. Ez a fajta black-box megközelítés lehetőséget adott arra, hogy az alkalmazás működését külső szemmel, előzetes technikai tudás nélkül vizsgálják meg. A visszajelzések alapján finomítottam a kezelőfelületet, kijavítottam a használat során felmerülő hibákat, és megerősítést nyert, hogy az alkalmazás az elvárt funkcionalitással rendelkezik.

Összességében a háromféle módszer alkalmazása – fejlesztői manuális tesztelés, automatizált egységtesztek és felhasználói validáció – olyan átfogó tesztelési keretrendszert eredményezett, amely biztosította a rendszer megbízhatóságát, funkcionalitását és használhatóságát egyaránt. Ez a megközelítés hozzájárult ahhoz, hogy az elkészült webalkalmazás éles környezetben is stabilan és hiba nélkül működjön.

\subsection{xUnit alapú automatizált tesztelés}

Az ASP.NET alkalmazásokhoz jól illeszthető xUnit keretrendszert használtam az egységtesztek megírására, amely egy modern, könnyen kezelhető és széles körben támogatott tesztelési eszköz. A fejlesztőkörnyezeten belül egyszerűen integrálható, és támogatja a tesztek paraméterezését, aszinkron függvények kezelését, valamint a mock objektumokkal való tesztelést is.

A projekt során külön egységteszt-projektet hoztam létre, ahol minden kontroller osztály külön fájlban szerepel, és a megfelelő metódusok lefedésére külön-külön teszteseteket írtam. A StopController példáján keresztül bemutatom, hogyan valósult meg egy ilyen tesztelés.

Az xUnit keretrendszer lehetővé teszi, hogy minden tesztfüggvényt [Fact] attribútummal lássunk el, amely jelzi, hogy az adott metódus egy önálló tesztesetként futtatható. Az egyes metódusok végén az Assert osztály segítségével ellenőrizzük az elvárt eredményeket. A sikeres működés kritériuma, hogy az alkalmazás a megfelelő típusú válasszal térjen vissza (például ViewResult vagy RedirectToActionResult), és hogy az adatok helyesen kerüljenek a modellbe.

A teszteléshez In-Memory adatbázist használtam, amely lehetővé teszi, hogy az Entity Framework teljes értékűen működjön anélkül, hogy tényleges fizikai adatbázist kellene létrehozni. Ez a módszer gyors, megbízható, és kiválóan alkalmas az izolált, reprodukálható tesztelési környezet biztosítására.

A következő példában a CreateControllerWithData metódus segítségével egy StopsController példányt hozunk létre előre feltöltött adatokkal. Az InMemoryDatabase minden tesztnél új, egyedi adatbázist hoz létre, így az adatok nem keverednek.

\begin{lstlisting}
private StopsController CreateControllerWithData(List<Stop> seedData)
{
var options = new DbContextOptionsBuilder<AppDbContext>()
.UseInMemoryDatabase(Guid.NewGuid().ToString())
.Options;

```
var context = new AppDbContext(options);
context.Stops.AddRange(seedData);
context.SaveChanges();

var config = new Mock<IConfiguration>();
var logger = new Mock<ILogger<AdminMessage>>();

return new StopsController(context, config.Object, logger.Object);
```

}
\end{lstlisting}


A következő teszteset célja, hogy ellenőrizze a StopsController osztály Index metódusának helyes működését. A teszt során először egy StopsController példány kerül létrehozásra egy segédfüggvényen keresztül, amelyet két előre definiált Stop objektummal inicializálunk. Ez az adatsor egy memóriában létrehozott adatbázisba kerül, amely lehetővé teszi a teszt izolált és reprodukálható végrehajtását, valós adatbázis használata nélkül.

A controller.Index() metódus meghívása után a visszatérési értéket a ViewResult típus ellenőrzésével validáljuk, mivel az Index metódusnak nézetet kell visszaadnia a megállók listájával. A modell típusának vizsgálata során ellenőrizzük, hogy az valóban IEnumerable<Stop> típusú, amely biztosítja, hogy a nézet megkapja a megfelelő adatszerkezetet. Végül a model.Count() metódussal azt is ellenőrizzük, hogy a nézet pontosan két Stop objektumot kapott, ami megegyezik a teszt során inicializált lista elemszámával.

Ez a teszt tehát teljes körűen lefedi az Index metódus működését: vizsgálja a visszatérési típus helyességét, az átadott modell struktúráját, valamint az adatok pontosságát is. A teszt sikeres lefutása garantálja, hogy az Index akció megbízhatóan működik, és hibamentesen biztosítja az összes megálló listázását a felhasználói felület számára.

\begin{lstlisting}
\[Fact]
public void Index_ReturnsViewResult_WithAllStops()
{
var controller = CreateControllerWithData(new List<Stop>
{
new Stop { StopId = "1", StopName = "Stop A" },
new Stop { StopId = "2", StopName = "Stop B" }
});

var result = controller.Index();

var viewResult = Assert.IsType<ViewResult>(result);
var model = Assert.IsAssignableFrom<IEnumerable<Stop>>(viewResult.Model);
Assert.Equal(2, model.Count());
}
\end{lstlisting}

A UserDetail\_ReturnsStop\_WhenValidId nevű teszteset célja annak ellenőrzése, hogy a StopsController osztály UserDetail metódusa helyesen működik-e érvényes azonosító megadása esetén. A teszt során először létrehozunk egy kontrollert, amelyet egy előre definiált, egyetlen Stop entitást tartalmazó memóriabeli adatbázissal inicializálunk. Ez biztosítja, hogy a teszt környezet független legyen az éles adatbázistól, és pontosan kontrollálható legyen a bemeneti adat.

A controller.UserDetail("1") hívás egy aszinkron metódus. A visszaadott eredménynek ViewResult típusúnak kell lennie, mivel a UserDetail metódus egy nézetet ad vissza, amely tartalmazza a lekérdezett megálló adatait. Ezt az Assert.IsType<ViewResult> segítségével ellenőrizzük.

A lekérdezés eredményeként visszakapott modell típusát is validáljuk, megvizsgáljuk, hogy az valóban Stop típusú objektum-e, majd az Assert.Equal segítségével összevetjük a lekérdezett megálló nevét az eredetileg definiált "Stop A" értékkel. Ez biztosítja, hogy az adatok nemcsak a megfelelő típusban, hanem a várt tartalommal is kerülnek visszaadásra a nézet számára.

\begin{lstlisting}
\[Fact]
public async Task UserDetail_ReturnsStop_WhenValidId()
{
var controller = CreateControllerWithData(new List<Stop>
{
new Stop { StopId = "1", StopName = "Stop A" }
});

var result = await controller.UserDetail("1");

var viewResult = Assert.IsType<ViewResult>(result);
var stop = Assert.IsType<Stop>(viewResult.Model);
Assert.Equal("Stop A", stop.StopName);
}
\end{lstlisting}

A Create\_AddsStop\_AndRedirects nevű függvény célja annak igazolása, hogy a StopsController Create metódusa helyesen hoz létre új adatbázisrekordot, valamint hogy az adatok mentése után a felhasználót a megfelelő oldalra irányítja vissza.

A teszt során először egy memóriában futó adatbázist hozunk létre az Entity Framework beépített InMemoryDatabase szolgáltatásával.  Az AppDbContext példányosítása után létrehozzuk a StopsController egy példányát, amelyhez mockolt IConfiguration és ILogger objektumokat rendelünk, így minimalizálva a külső függőségeket.

A teszteset fő része az új Stop objektum létrehozása, amelyhez konkrét értékeket adunk meg, például név, szélességi és hosszúsági koordináták. Ezt követően meghívjuk a Create metódust, és await kulcsszóval aszinkron módon megvárjuk a végrehajtását.

A válaszként kapott eredményt RedirectToActionResult típusra ellenőrizzük, ami azt jelzi, hogy az új rekord sikeres mentését követően az alkalmazás átirányítja a felhasználót az Index nézetre. A context.Stops kollekció lekérdezésével validáljuk, hogy pontosan egy rekord került be az adatbázisba, és az új elem neve valóban megegyezik a megadott „New Stop” értékkel.

Ez a teszteset kulcsfontosságú abból a szempontból, hogy megerősíti a Create művelet működését: a kontrolleren belüli logika nemcsak a helyes viselkedést produkálja, hanem az adatok mentése is megbízható módon történik. Az ilyen típusú validáció hozzájárul az alkalmazás robusztusságához és stabil működéséhez különféle felhasználói interakciók során.

\begin{lstlisting}
\[Fact]
public async Task Create\_AddsStop\_AndRedirects()
{
var options = new DbContextOptionsBuilder<AppDbContext>()
.UseInMemoryDatabase(Guid.NewGuid().ToString())
.Options;

using var context = new AppDbContext(options);
var controller = new StopsController(context, new Mock<IConfiguration>().Object, new Mock<ILogger<AdminMessage>>().Object);

var stop = new Stop { StopName = "New Stop", Latitude = 3, Longitude = 4 };

var result = await controller.Create(stop);

var redirect = Assert.IsType<RedirectToActionResult>(result);
Assert.Equal("Index", redirect.ActionName);
Assert.Single(context.Stops);
Assert.Equal("New Stop", context.Stops.First().StopName);
}
\end{lstlisting}

A Modify\_UpdatesStop\_WhenValidData nevű teszteset célja annak biztosítása, hogy a meglévő Stop entitás módosítása megfelelő módon megtörténjen, és a művelet után a rendszer megfelelő átirányítást hajtson végre az Index nézet irányába. A Modify metódus működése szoros párhuzamban áll a Create metódus logikájával.

Míg a Create egy teljesen új rekord hozzáadását végzi el, addig a Modify már létező adatok frissítésére szolgál.A teszt során a StopName mezőt egy új értékre írjuk át, majd meghívjuk a Modify metódust az azonosító és a módosított objektum átadásával. A Modify belső működésében először ellenőrzi, hogy az id egyezik-e az entitás azonosítójával, majd az Update és SaveChangesAsync metódusok segítségével véglegesíti a módosításokat. A sikeres végrehajtás után átirányítja a felhasználót az Index nézetre, amit a tesztben az Assert.IsType<RedirectToActionResult> és Assert.Equal("Index", ...) utasításokkal validálunk.

A Create metódussal szemben itt nem új rekord keletkezik, hanem egy meglévő objektum mezői frissülnek, így a teszt fókusza is más: míg a Create esetén az adat tényleges bekerülése a cél, addig a Modify esetén a meglévő adat sikeres frissítése, és ezzel együtt az esetleges ütközések kezelése válik fontossá.

\begin{lstlisting}
\[Fact]
public async Task Modify\_UpdatesStop\_WhenValidData()
{
var stopId = Guid.NewGuid().ToString();
var stop = new Stop { StopId = stopId, StopName = "Old Name" };
var controller = CreateControllerWithData(new List<Stop> { stop });

```
stop.StopName = "Updated Name";
var result = await controller.Modify(stopId, stop);

var redirect = Assert.IsType<RedirectToActionResult>(result);
Assert.Equal("Index", redirect.ActionName);
```

}
\end{lstlisting}

 Végezetül, a törlés tesztelésénél azt figyeljük meg, hogy az adott rekord eltávolításra kerül-e, és hogy a rendszer visszatér-e az entitások listanézetéhez.

A DeleteConfirmed\_RemovesStop\_WhenValidId elnevezésű teszteset elsődleges célja annak validálása, hogy a törlésre szánt Stop entitás valóban eltávolításra kerül az adatbázisból, és a művelet végén a felhasználó a megfelelő nézetre — jelen esetben az Index listanézetre — kerül visszairányításra.

A metódus felépítése hasonló a létrehozási (Create) tesztesethez abban az értelemben, hogy itt is InMemory adatbázist használunk a DbContext példány elkülönítésére. Az eltérés azonban a teszt céljában rejlik: nem új adat hozzáadása történik, hanem egy meglévő entitás eltávolítása.

A tesztben először létrehozunk egy egyedi azonosítóval rendelkező megállót (Stop objektumot), amelyet előzetesen beillesztünk az adatbázisba. Ez biztosítja, hogy a törléshez legyen releváns tesztadat. A DeleteConfirmed metódus meghívásakor átadjuk ezt az azonosítót, és a visszatérési érték típusát ellenőrizzük — ez egy RedirectToActionResult, amely az Index akcióra irányítja vissza a felhasználót.

Ellentétben a Create teszttel, ahol azt vizsgáltuk, hogy az új adat sikeresen hozzáadódott, itt a validáció arra irányul, hogy a rekord valóban eltűnt az adatbázisból. Ez a fajta teszt különösen hasznos olyan esetekben, ahol a törlési művelet érzékeny vagy visszafordíthatatlan hatású lehet — például ha az adott entitás kapcsolatban áll más rekordokkal is. A törlés során alkalmazott ellenőrzések, mint például a meglét validálása (FindAsync) vagy a jogosultságok vizsgálata, külön tesztesetekkel tovább bővíthetők, de az alapművelet helyességének vizsgálatát ez az egység jól reprezentálja.

\subsection{Felhasználói visszajelzéseken alapuló kézi tesztelés}

Az automatizált unit tesztek mellett fontosnak tartottam a kézi tesztelést is, különös tekintettel a valós felhasználói élményre és a felület gyakorlati használhatóságára. Ennek érdekében a projekt egyes fázisainak lezárása után több ismerősömet is bevontam a tesztelési folyamatba, akik eltérő informatikai háttérrel rendelkeznek. A cél az volt, hogy minél szélesebb kört lefedve kapjak visszajelzést arról, hogyan értelmezik és használják az alkalmazást különböző gondolkodásmódú emberek.

A tesztelés minden esetben az én felügyeletem mellett zajlott, rövid szóbeli bevezetőt követően. Ismertettem a rendszer alapvető működését, majd átadtam számukra az irányítást, hogy szabadon kipróbálhassanak bizonyos funkciókat – például megállók megtekintése, jegyvásárlás, bejelentkezés vagy térképes nézetek használata. Fontos szempont volt, hogy ne irányítsam őket aktívan, csupán megfigyeljem, miként boldogulnak önállóan a rendszer használatával.

Ez a fajta manuális tesztelés különösen fontos volt abból a szempontból, hogy az én fejlesztői látásmódom eltérhet a végfelhasználókétól. Míg én ismerem az alkalmazás szerkezetét és logikáját, számomra magától értetődő lehet olyan funkciók használata, amelyek egy új felhasználó számára kevésbé egyértelműek. Éppen ezért ezek a visszacsatolások fontos szerepet töltöttek be a felhasználói élmény optimalizálásában. A továbbiakban pár felhasználó személyes véleményét mutatom be.

\subsection{Felhasználói visszajelzések, észrevételek}

\subsubsection{Szőcs Tamás - Kolozsvár}
\paragraph*{1. Milyennek látja az aktuális tömegközlekedési applikációkat?} 
A legtöbb app-nak túlkomplikált a user interface-je, nem elég letisztult, kevés infó van feltűntetve. Az SDMNG-n viszont könnyű volt a navigálás, minden szembe ugrik.

\paragraph*{2. Mi az ami megnehezíti a használatukat, ha van ilyen?} 
Nem a legpontosabbak, jegyeket nem lehet vásárolni rajtuk keresztül (tudtom szerint), tele vannak reklámokkal.

\paragraph*{3. Milyennek találta az SDMNG webalkalmazást? Könnyű volt használni, navigálni, érthető?} 
Igen, az interface letisztult, minden ott van ahol lennie kell (logikus).

\paragraph*{4. Miben lehetne kibővíteni?}
Designosabbá tenni (tudom, állavizsga projekt, de más nem ugrott be)

\paragraph*{5. Milyennek tűnik az alkalmazás felülete, tiszta, átlátható, kedvez a szemnek?}
Az egyik legjobb, legtisztább és REKLÁM MENTES felület amit valaha használtam, jól össze rakott és logikusan elhelyezett gombokkal.

\subsubsection{Bakó Boglárka - Marosvásárhely, Kolozsvár}

\paragraph*{Milyennek látja az aktuális tömegközlekedési applikációtat?}
A legtöbb alkalmazás amit használtam ezekben kritizálható:
\begin{itemize}
    \item Nincs térképes megjelenítés minden vonalhoz.
    \item Egy analóg alkalmazás áll rendelkezésre, amely nagyon általánosan tükrözi a valóságot.
    \item Nincsenek rendszeres frissítések.
    \item Nem található meg könnyen minden olyan információ, amely a felhasználók számára fontos lehet.
    \item A felület nem logikus; a funkciók között keresgélni kell.
    \item Ugyanaz a funkció több, nem releváns helyen is megjelenik – például: más vonalak szerepelnek ugyanazon vonal nézetében, de nem teljeskörűen, csak részlegesen.
    \item Lassú betöltés, hosszú várakozási idő.
    \item Túl sok információ jelenik meg egyszerre – a felhasználó könnyen elveszik benne, és nem tudja, mit is kellene keresnie.
    \item Felesleges adatokat kér az alkalmazás.
    \item Bizonyos esetekben túl bonyolult a regisztráció vagy a jegyfoglalás folyamata.
\end{itemize}

\paragraph*{Mi az, ami megnehezítheti a használatát, ha van ilyen?}
\begin{itemize}
    \item Ha nincs olyan opció, hogy az üzenetem el lett küldve, és x idő alatt választ kapok, az bizonytalanságot kelthet a felhasználóban.
    \item A mobilapplikáción való megjelenítés vagy áttelepítés.
\end{itemize}

\paragraph*{Milyennek találta az SDMAG webalkalmazást? Könnyű volt használni, navigálni, érthető?}
 A webalkalmazást nagyon könnyű használni, könnyen lehet navigálni, nagyon érthető és átlátható. Nagyon értékes funkciónak tartom azt, hogy üzenni lehet nekik, mert így ha kérdésem van, elakadtam vagy reklamálni szeretnék, akkor van rá lehetőség. Fontosnak tartom azt a funkciót is, hogy meg lehet nézni, hány jegy van még elérhető, mert így tudom, hogy siessek a lefoglalással, vagy várjam meg a következőt.

\paragraph*{Miben lehetne kibővíteni az alkalmazást?}

A rendszer az alábbi funkciókkal bővíthető lenne:
\begin{itemize}
    \item Minden járatnál jelenjen meg az ár.
    \item Legyen lehetőség ülőhely-foglalásra, vizuális választással.
    \item Valós idejű helymeghatározás: a busz helyzete és késése látható legyen.
    \item QR-kódos jegy fogadása mobiltelefonra vagy más digitális eszközre.
    \item Push értesítés vagy e-mail, ha menetrendváltozás vagy késés történik.
    \item Értesítés a járat indulása előtt 10–15 perccel.
    \item Értékelési lehetőség 5 csillagos rendszerben.
    \item Többnyelvű applikáció.
    \item Kedvencek lista: rendszeres járatok gyors elérése.
\end{itemize}

\paragraph*{Milyennek tűnik az alkalmazás felülete? Tiszta, átlátható, kedvez a szemnek?}
Nagyon átlátható és logikus, vizuálisan kellemes és felesleges gombok nélküli felületet nyújt. Ha ránézek az applikációra, azonnal látom a térképet, útvonalakat, és a járatokhoz tartozó szabad jegyek számát – ez gyors döntéshozatalt segít. Minden fő információ könnyen hozzáférhető. Összességében az a

\subsubsection{Szabó Róbert - Kolozsvár}

\paragraph*{Milyennek látja az aktuális tömegközlekedési applikációkat?}
Útvonaltervezés szempontjából:
A legtöbb applikációt a mai napig veri a google maps, ez sokkal elérhetőbb, legtöbb nagy városban kíválóan működik, nem kell minden városhoz külön appot letölteni emiatt.
A google maps hátrányai a városok közötti, romániában privát cégek által biztosított járatok, amelyeket kevésbé ismer ezáltal nehezebb az útvonaltervezés. (alapjáraton a tömegközlekedés egy rakás szar külföldhöz képest)

Városi tömegközlekedés: 
Sok esetben a helyi vállalat saját applikációi főként a helyieknek szólnak ezért casual utazóként/turista ként nehezebb használni őket.
Városok közötti tömegközlekedés:
Az autogari.ro szerintem nagyon jó applikáció, ki lehet simán keresni, hogy honnan hova hánykor indul mikor érkezik stb, ahányszor használtam meg voltam elégedve vele.

\paragraph*{Mi az, ami megnehezítheti a használatát, ha van ilyen?}

Legtöbb esetben amikor ezen applikációkat használtam turista voltam, ezért gondot jelentett, hogy nem ismertem a várost. Páldául napi bérlet esetében van, hogy zónákra van felosztva a város, mint turista ha nem mutatja az applikáció, akkor nincs honnan tudjad, hogy a szállásod az most melyik zónához tartozik.
Vizuális ember vagyok ezért nekem a legtöbb esetben a város térkép nagyon sokat tudna segíteni, viszont sok esetben csak a megállók sorrendjét tüntetik fel.

\paragraph*{Milyennek találta a az SDMAG webalkalmazást? Könnyű volt használni, navigálni, érthető? }


Maga a felhasználói része teljesen user-friendly volt,  magától érhetődő, egyszerű de nagyszerű.

Miben lehetne kibővíteni?

Szerintem önmagában ez az applikáció nagyon jó. Ha ezt egy tömegközlekedési vállalat felvásárolná, akkor ebben az estben szükségesek lennének olyan bővítések amik specifikusak ennek a vállat által nyújtott szolgátatásoknak. Pld átszállási lehetőségek, buszon nyújtott szolgáltatások feltüntetése (wifi, hideg víz stb)

\paragraph*{Milyennek tűnik az alkalmazás felülete, tiszta, átlátható, kedvez a szemnek? }

(9/10) Tiszta és átlátható viszont kis dizájner segítség nem ártana amivel kicsit jobban feldobná. Az egész projekt nagyon szép és vagány, amire kell arra tökéletes.

\subsubsection{Király Zsolt - Felvinc, Kolozsvár}

\paragraph*{Milyennek látja az aktuális tömegközlekedési applikációkat?}

A mai tömegközlekedési rendszerek alapvetően jól átgondolt elvek mentén lettek megtervezve, azonban a gyakorlati kivitelezésük nem minden esetben sikerült megfelelően. Gyakori probléma, hogy az utasok számára releváns információk szétszórva, több különböző felületen érhetők el, és ezek gyakran csak hivatkozásokon keresztül vannak összekötve, ha ezek egyáltalán elérhetők. Ez jelentősen megnehezíti a tájékozódást, különösen új vagy idegen környezetben közlekedő felhasználók számára. Bár egyes alkalmazások átláthatóak és jól használhatók, az információk hiányos vagy nehézkes elérése gyakran csökkenti a felhasználói élményt.

\paragraph*{Milyennek találta az SDMAG webalkalmazást? Könnyű volt használni, navigálni, érthető?}

A webalkalmazást könnyű használni, logikusan van összerakva, könnyen lehet jegyet vásárolni, valós időben lehet követni a jegyek elérhetőségét és az útvonalak változását. Nagyon fontosnak tartom azt is, hogy a buszok megtalálását rendszám információval könnyű azonosítani.

\paragraph*{Mi az, ami megnehezítheti a használatát, ha van ilyen?}

A használatát megnehezítheti az, ha nem lehetne követni az üzeneteimre kapott válaszokat, vagy ha az újabb frissítéseket és információkat nem lehetne reális időben nyomon követni.

\paragraph*{Miben lehetne kibővíteni?}

Kibővíteni véleményem szerint azzal lehetne, ha az ülőhelyet ki lehetne választani a buszon, illetve élő időbeli értesítések jelennének meg, ha változik a menetrend vagy az indulási helyszín.

\paragraph*{Milyennek tűnik az alkalmazás felülete, tiszta, átlátható, kedvez a szemnek?}

Az alkalmazás felülete szerintem könnyen átlátható, nincsenek zavaró tényezők, amelyek elterelhetnék a figyelmet. Véleményem szerint nagy segítség, hogy a menetrend útvonala is fel van tüntetve. Minden fontos információ elérhető a webalkalmazáson.
