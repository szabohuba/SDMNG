\section{Projekt telepítése Azure Web App és SQL Server környezetbe}

A dolgozatom során egy ASP.NET MVC alapú webalkalmazást fejlesztettem ki, amelyhez relációs adatbázis-kezelőként az SQL Server rendszert használtam. A projekt elkészülte után egyértelművé vált számomra, hogy a működő alkalmazást érdemes lenne valós környezetben is kipróbálni, így esett a választásom a Microsoft Azure felhőalapú szolgáltatásaira. Az Azure több szempontból is ideális választásnak bizonyult: gyorsan és egyszerűen teszi lehetővé a webalkalmazások hosztolását, valamint kiválóan integrálódik a Microsoft fejlesztői ökoszisztémájába, beleértve a Visual Studio-t, az SQL Server-t is. 

Első lépésként szükség volt egy SQL Server típusú relációs adatbázis létrehozására a felhőben, amelyhez az Azure Portal felületét használtam. A portálon belépés után a bal oldali menüből kiválasztottam az \textit{SQL databases} opciót, majd új adatbázist hoztam létre. A folyamat során megadtam egy egyedi adatbázisnevet, kiválasztottam az előfizetésemet és egy, az adatbázist kiszolgáló SQL szervert. Amennyiben nem rendelkeztem még szerverrel, azonnal lehetőségem nyílt új szerver létrehozására, amelyhez meg kellett adnom egy egyedi szervernevet, valamint az adminisztrátori felhasználónevet és jelszót. Fontos volt, hogy a szervert egy számomra stabil régióban hozzam létre a késleltetés minimalizálása érdekében.

Miután a szerver és az adatbázis létrejött, a következő lépésként az eddig lokálisan használt adatbázisomat kellett áthelyeznem az újonnan létrehozott Azure SQL környezetbe. Ehhez az SQL Server Management Studio (SSMS) alkalmazást használtam, amelyen keresztül közvetlenül kapcsolódni tudtam az Azure-ban létrehozott SQL szerverhez. A helyi adatbázis exportálását követően az adatokat és a sémákat sikeresen importáltam a felhőalapú adatbázisba. Az adatok feltöltése után a Visual Studio-ban futó ASP.NET MVC alkalmazásom konfigurációs fájljában (appsettings.json) frissítettem az adatbáziskapcsolati karakterláncot (connection string), hogy az már az Azure SQL példányra mutasson. A kapcsolat tesztelését követően meggyőződtem arról, hogy az alkalmazás megfelelően képes kommunikálni a távoli adatbázissal. Ezzel a módosítással az alkalmazásom már nem a lokálisan tárolt adatokat használta, hanem a felhőben működő Azure SQL adatbázis adatait, amelyeket a továbbiakban minden adatkezelési művelethez igénybe vett.

A háttérinfrastruktúra előkészítése után következhetett maga a webalkalmazás éles környezetbe történő telepítése. Ehhez a Visual Studio \textit{Publish} funkcióját használtam, amely lehetőséget ad az alkalmazás közvetlen Azure Web App-re történő közzétételére. A megjelenő varázslóban új App Service példányt hoztam létre, amelynek során kiválasztottam az alkalmazás nevét, az előfizetést, a tárhelyet biztosító App Service Plan-t, valamint az Azure régiót. A rendszer automatikusan létrehozta az alkalmazás futtatásához szükséges környezetet.



