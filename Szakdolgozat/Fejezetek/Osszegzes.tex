\section*{Összegzés}


A projekt megvalósítása során számos hasznos tapasztalattal és gyakorlati tudással gazdagodtam, mind a tervezés, fejlesztés, dokumentálás, valamint a tesztelés során. Külön örömöt jelentett számomra, hogy az alkalmazás teljes életciklusát végigkísérhettem: az ötlet kidolgozásától kezdve a megvalósításon és telepítésen át egészen a dokumentáció elkészítéséig. A fejlesztés alatt nemcsak technikai ismereteim mélyültek el, hanem rendszerszintű rálátást is szereztem az alkalmazásfejlesztés komplex folyamatára.

A projekt személyes jelentőséggel is bírt számomra, hiszen rendszeres tömegközlekedési felhasználóként saját tapasztalataimból kiindulva dolgozhattam ki egy olyan megoldást, amely valódi, hétköznapi problémára kínál választ. A közlekedés átláthatóságával és információhiányával kapcsolatos nehézségek számomra is ismerősek voltak, így különösen motiváló élmény volt egy olyan rendszer fejlesztése, amely ezek orvoslását célozza. Ennek köszönhetően a projekt nem csupán egy fejlesztési feladatként, hanem egy személyes igényre épülő, értékes kezdeményezésként is szolgált.

A .NET környezet rugalmassága és bővíthetősége szintén hozzájárult a pozitív tapasztalatokhoz. Kiemelten tetszett, hogy a különböző külső könyvtárak – mint például a Leaflet térképes vizualizációhoz vagy a QR-kód generálás – egyszerűen és hatékonyan integrálhatók voltak a rendszerbe. Ezek az eszközök nemcsak funkcionálisan egészítették ki az alkalmazást, hanem jelentősen javították a felhasználói élményt és a vizuális megjelenést is.

A tesztelés szintén fontos része volt a folyamatnak: az eszközök letisztult szerkezete és célratörő működése lehetővé tette a rendszer átlátható, hatékony ellenőrzését mind fejlesztői, mind felhasználói oldalról. A problémák azonosítása és a megoldások kidolgozása nemcsak technikai fejlődést hozott, hanem egyben sikerélményt is nyújtott minden egyes leküzdött akadály után.

Összességében úgy érzem, hogy a projekt minden egyes szakasza hozzájárult szakmai fejlődésemhez, elmélyítette érdeklődésemet a szoftverfejlesztés iránt, és megerősítette bennem azt az érzést, hogy a fejlesztés nemcsak kihívásokat, hanem valódi örömöt és alkotói szabadságot is jelent.

