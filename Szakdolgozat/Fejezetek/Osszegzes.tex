\section{Összegzés}

Összegzés képpen a projekt megvalósítása során számos hasznos tapasztalattal és gyakorlati tudással gyarapodtam, mind a tervezés, fejlesztés, dokumentálás, valamint a tesztelés során. Külön élményt jelentett számomra, hogy az alkalmazás teljes életciklusát végigkísérhettem: az ötlet kidolgozásától kezdve a megvalósításon és telepítésen át egészen a dokumentáció elkészítéséig. A fejlesztés alatt nemcsak technikai ismereteim mélyültek el, hanem rendszerszintű rálátást is szereztem az alkalmazásfejlesztés komplex folyamatára.

A dolgozat kiindulópontját egy valós, a mindennapokat érintő probléma jelentette: a közösségi közlekedési rendszerek információhiánya és átláthatatlansága. Ennek ismeretében célom egy olyan működő, gyakorlati haszonnal bíró rendszer fejlesztése volt, amely korszerű, letisztult megközelítést kínál a jelenleg használt megoldásokkal szemben. A projekt alapját a szoftvertervezés ismert mintái adták, különösen az MVC architektúra, amely segített a rendszer moduláris, jól strukturált kialakításában. Ennek révén nemcsak a fejlesztés vált átláthatóbbá, hanem a jövőbeni bővítés és karbantartás is könnyen megvalósíthatóvá vált.

A fejlesztési folyamat során sikerült saját elképzelésem alapján létrehozni egy olyan alkalmazást, amely a valós közlekedési problémákra kínál konkrét megoldást. A rendszer alapvető működési logikáját, felépítését és vizuális megjelenését is magam határoztam meg, és ezek a felhasználói tesztek során visszaigazolást is nyertek. A kipróbálók visszajelzései megerősítették, hogy az alkalmazás intuitív, könnyen használható és a célját sikeresen betölti.

A .NET keretrendszerrel való munka során jelentős mértékben elmélyítettem tudásomat mind az architekturális, mind a gyakorlati oldalon. Különösen értékes tapasztalatokat szereztem az ASP.NET MVC modell működéséről, az Identity rendszer integrációjáról, az adatbáziskezelésről az Entity Framework segítségével, valamint a külső szolgáltatások – mint a QR-kód generálás vagy az SMTP-alapú emailküldés – alkalmazásáról. Ezek a komponensek nemcsak bővítették a rendszer funkcionalitását, de hozzájárultak a kód újrahasznosíthatóságához és tesztelhetőségéhez is.

A projekt során sikerült teljes körűen bejárni a szoftverfejlesztési életciklus szinte minden szakaszát: az igényfelméréstől kezdve a tervezésen és implementáláson át egészen a tesztelésig és kiértékelésig. Ez a gyakorlat nem csupán elméleti rálátásom mélyítette el, hanem hozzájárult ahhoz is, hogy átlássam, hogyan lehet egy kezdeti ötletből működő alkalmazást létrehozni. 

A tesztelési szakasz során kiemelt figyelmet fordítottam arra, hogy a rendszer működése mind technikai, mind felhasználói szempontból megbízható legyen. A tesztelés során szerzett hibajavítási tapasztalatok és a pozitív visszajelzések tovább erősítették bennem azt az érzést, hogy egy értékes és jól működő szoftver valósúlt meg.

Összegzésképpen elmondható, hogy ez a projekt nem csupán egy fejlesztési feladat volt, hanem egy személyes jelentőséggel bíró kezdeményezés, amelynek során lehetőségem nyílt egy olyan problémára választ adni, amely a hétköznapjaim része. Úgy érzem, hogy az általam javasolt megoldás megalapozott, jól strukturált, és a visszajelzések alapján felhasználóbarát módon valósult meg. A megszerzett tapasztalatok és az elért eredmények megerősítettek abban, hogy a szoftverfejlesztés nemcsak technikai kihívások sorozata, hanem egy kreatív, inspiráló folyamat is, amely valós értéket teremthet.
